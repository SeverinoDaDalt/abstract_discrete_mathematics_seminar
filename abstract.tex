%! Author = sdadalt
%! Date = 20/3/25

% Preamble
\documentclass[11pt]{article}

% Packages
\usepackage{amsmath}
\usepackage{amsthm}
\usepackage{mathtools}
\usepackage{amsfonts}
\usepackage{amssymb}
\usepackage{tikz-cd}
\usepackage{wasysym}
\DeclareSymbolFont{matha}{OML}{txmi}{m}{it}% txfonts
\DeclareMathSymbol{\varv}{\mathord}{matha}{118}

% Title
\title{Seminar on Algebra, Geometry and Discrete Mathematics}
\author{Severino Da Dalt}

\newtheorem{theorem}{Theorem}[section]
\newtheorem{corollary}{Corollary}[theorem]
\newtheorem{lemma}[theorem]{Lemma}
\newtheorem{prop}[theorem]{Proposition}

\theoremstyle{definition}
\newtheorem{defn}[theorem]{Definition}
\newtheorem{rk}[theorem]{Remark}

% Document
\begin{document}

    \maketitle

    \section{Proof of asymptotic behaviour of eigenvalues in families of expanders}\label{sec:section}
    The main topic of the seminar is the study of families of expanders in the context of connected, $k$-regular and finite graphs.
    A family of expanders consists in a sequence of graphs of increasing size such that the isoperimetric number $h(X)$ satisfies $h(X) \geq \epsilon$
    for some fixed positive $\epsilon$ and for all graphs $X$ in the sequence.
    The isoperimetric number, in the context of finite graphs, can be defined as:
    $$h(x) = \min{ \left\{ \frac{|\partial F|}{|F|} \mid F \subseteq V, |F| \leq \frac{n}{2}\right\} }$$
    %\newline

    Earlier in the seminar, we established that the isoperimetric number is bounded by the spectral gap, in particular,
    $$\frac{k - \mu_1}{2} \leq h(x) \leq \sqrt{2k(k - \mu_1)}$$
    where $\mu_1$ denotes the first nonzero eigenvalue.
    Moreover, we showed that the spectral gap cannot grow too much, i.e.
    $$\liminf\limits_{m \rightarrow +\infty}\mu_1 \geq 2 \sqrt{k-1}$$
    %\newline

    In this session, we aim to strengthen this result by proving that a non-trivial fraction of the eigenvalues of the
    graph lies within the interval $[(2 - \epsilon) \sqrt{k-1}, k]$ for any given $\epsilon > 0$.
    \newline

    To prove this, we introduce a set of matrices $A_r$, which are polynomials in the incidence matrix $A$.
    The entries of $A_r$ describe the number of backtracking-free paths of length $r$ between two vertices.
    \newline

    After establishing the necessary properties of these polynomials, we compute their generating function which
    is closely related to the generating function of Chebyshev polynomials $(U_m)_{m\in \mathbb{N}}$.
    This relationship between the two polynomials leads to the following key result:
    \begin{equation}
        \label{eq:1}
        \sum_{x \in V} \sum_{0\leq r \leq\frac{m}{2}} \left(A_{m-2r}\right)_{xx} = (k-1)^{\frac{m}{2}} \sum_{j=0}^{n-1} U_m\left(\frac{\mu_j}{2\sqrt{k-1}}\right)
    \end{equation}
    for all $m \in \mathbb{N}$.
    \newline

    We then introduce the measure:
    $$\varv = \frac{1}{n} \sum^{n-1}_{j=0} \delta_{\frac{\mu_j}{\sqrt{k-1}}}$$
    where $\delta_a$ is the Dirac measure at $a$, and $\mu_0, \dots, \mu_{n-1}$ are the eigenvalues of the graph.
    The Dirac measure $\delta_a$ acts as an indicator function of whether $a$ lies in a given interval.
    Hence, this measure is in some way counting the number of eigenvalues within a specific range, which is central to the argument.
    \newline

    Using equation~\eqref{eq:1}, we show that, for an appropriate choice of $L \geq 2$, $\varv$ satisfies:
    $$\int_{-L}^L U_m\left(\frac{x}{2}\right) d\varv \geq 0$$
    for all $m$.
    Finally, we will see that this condition is sufficient to prove that the measure $\varv$ has a positive support in the interval $[2-\epsilon, L]$.
    In other words,
    $$\varv[2-\epsilon, L] \geq C$$
    where $C$ is a constant that only depends on $\epsilon$ and $L$, and is strictly positive.
    This proves the main result.
    \newline

    Additionally, we show a similar conclusion for the negative eigenvalues of the graph, generalizing an earlier result.

\end{document}