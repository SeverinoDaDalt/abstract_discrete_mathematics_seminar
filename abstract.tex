%! Author = sdadalt
%! Date = 20/3/25

% Preamble
\documentclass[11pt]{article}

% Packages
\usepackage{amsmath}
\usepackage{amsthm}
\usepackage{mathtools}
\usepackage{amsfonts}
\usepackage{amssymb}
\usepackage{tikz-cd}
\usepackage{wasysym}
%These are my Commutative Algebra Teory notes from the course given by Prof. Josep Alvarez Montaner at UPC in 2023.

% Title
\title{Seminar on Algebra, Geometry and Discrete Mathematics}
\author{Severino Da Dalt}

\newtheorem{theorem}{Theorem}[section]
\newtheorem{corollary}{Corollary}[theorem]
\newtheorem{lemma}[theorem]{Lemma}
\newtheorem{prop}[theorem]{Proposition}

\theoremstyle{definition}
\newtheorem{defn}[theorem]{Definition}
\newtheorem{rk}[theorem]{Remark}

% Document
\begin{document}

    \maketitle

    \section{Proof of asymptotic behaviour of eigenvalues in families of expanders}\label{sec:section}

        \begin{rk}
            Unless otherwise specified, all rings we discuss will be commutative with unit.
        \end{rk}

        \begin{defn}
            Let $f: A \rightarrow B$ be a ring homomorphism, and let $J \subset B$ be an ideal.
            Then, the \emph{contraction} of $J$ to $A$ is $J^{c} \coloneqq \{a \in A \mid f(a) \in J\}$.
        \end{defn}

        \begin{prop}
            In the above situation, $J^{c}$ is an ideal of $A$.
        \end{prop}

            \begin{proof}
                It is an additive subgroup of $A$ as $f$ is an additive group homomorphism.
                Furthermore, if $a \in J^{c}$ and $r \in A$, then $f(a) \in J$ so $f(ra) = f(r)f(a) \in J$ as $J$ is an ideal.
            \end{proof}

\end{document}