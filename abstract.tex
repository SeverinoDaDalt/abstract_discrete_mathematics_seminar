%! Author = sdadalt
%! Date = 20/3/25

% Preamble
\documentclass[11pt]{article}

% Packages
\usepackage{amsmath}
\usepackage{amsthm}
\usepackage{mathtools}
\usepackage{amsfonts}
\usepackage{amssymb}
\usepackage{tikz-cd}
\usepackage{wasysym}
\setlength{\parindent}{0pt}
\usepackage[margin=70pt]{geometry}
\DeclareSymbolFont{matha}{OML}{txmi}{m}{it}% txfonts
\DeclareMathSymbol{\varv}{\mathord}{matha}{118}

% Title
%\title{Seminar on Algebra, Geometry and Discrete Mathematics}
%\author{Severino Da Dalt}

\newtheorem{theorem}{Theorem}[section]
\newtheorem{corollary}{Corollary}[theorem]
\newtheorem{lemma}[theorem]{Lemma}
\newtheorem{prop}[theorem]{Proposition}

\theoremstyle{definition}
\newtheorem{defn}[theorem]{Definition}
\newtheorem{rk}[theorem]{Remark}

% Document

\begin{document}

    {\Large Seminar on Algebra, Geometry and Discrete Mathematics -- MAMME}
    \newline
    {Severino Da Dalt -- 2025-03-28}

    \vspace{0.5cm}

    \hrule

    \vspace{0.5cm}

    \section{Proof of asymptotic behaviour of eigenvalues in families of expanders}\label{sec:section}
    The main topic of the seminar is the study of families of expanders in the context of connected, $k$-regular and finite graphs.
    A family of expanders consists in a sequence of graphs of increasing size such that the isoperimetric number $h(X)$ satisfies $h(X) \geq \epsilon$
    for some fixed positive $\epsilon$ and for all graphs $X$ in the sequence.
    %The isoperimetric number, in the context of finite graphs, can be defined as:
    %$$h(x) = \min{ \left\{ \frac{|\partial F|}{|F|} \mid F \subseteq V, |F| \leq \frac{n}{2}\right\} }$$
    \newline

    Earlier in the seminar, we established that the isoperimetric number is bounded by the spectral gap, in particular,
    $$\frac{k - \mu_1}{2} \leq h(x) \leq \sqrt{2k(k - \mu_1)}$$
    where $\mu_1$ denotes the first nonzero eigenvalue.
    Moreover, we showed that the spectral gap cannot grow too much, i.e.
    $$\liminf\limits_{m \rightarrow +\infty}\mu_1 \geq 2 \sqrt{k-1}$$
    %\newline

    In this session, we aim to strengthen this result by proving that a non-trivial fraction of the eigenvalues of the
    graph lies within the interval $[(2 - \epsilon) \sqrt{k-1}, k]$ for any given $\epsilon > 0$.
    \newline

    To prove this, we introduce a set of matrices $A_r$, which are polynomials in the incidence matrix $A$.
    The entries of $A_r$ describe the number of backtracking-free paths of length $r$ between two vertices.
    \newline

    After establishing the necessary properties of these polynomials, we compute their generating function, which is
    closely related to the generating function of Chebyshev polynomials $(U_m)_{m\in \mathbb{N}}$.
    This allows us to relate the two polynomials:
    \begin{equation}
        \label{eq:relation_T_U}
    T_m = (k-1)^{\frac{m}{2}} U_m\left(\frac{A}{2 \sqrt{k-1}}\right)
    \end{equation}
    where $T_m = \sum_{0 \leq r \leq \frac{m}{2}}$.
    We then compute the trace of $T_m$ by, on one side using their definition, and on the other using
    equation~\eqref{eq:relation_T_U}.
    It follows that
    \begin{equation}
        \label{eq:relation_A_U}
        \sum_{x \in V} \sum_{0\leq r \leq\frac{m}{2}} \left(A_{m-2r}\right)_{xx} = (k-1)^{\frac{m}{2}} \sum_{j=0}^{n-1} U_m\left(\frac{\mu_j}{2\sqrt{k-1}}\right)
    \end{equation}
    for all $m \in \mathbb{N}$.
    This result is particularly useful because it proves that the right side of the equation is positive.
    \newline

    We then introduce the measure:
    $$\varv = \frac{1}{n} \sum^{n-1}_{j=0} \delta_{\frac{\mu_j}{\sqrt{k-1}}}$$
    where $\delta_a$ is the Dirac measure at $a$, and $\mu_0, \dots, \mu_{n-1}$ are the eigenvalues of the graph.
    The Dirac measure $\delta_a$ acts as an indicator function of whether $a$ lies in a given interval.
    %Hence, this measure is in some way counting the number of eigenvalues within a specific range, which is central to the argument.
    Thus, $\varv$ satisfies:
    $$
    \varv[2-\epsilon, L] = \frac{1}{n} \sum_{j=0}^{n-1}
    $$
    \newline

    To conclude the proof we will use equation~\ref{eq:relation_A_U} to prove that $\varv$ measures at least a strictly
    positive amount $C = C(\epsilon)$ in the interval $[2-\epsilon, L]$.
    To prove this, we first define polynomials $X_m = U_m(\frac{x}{2})$ and $Y_m(x) = \frac{X_m(x)^2}{x - \alpha_m}$ where
    $\alpha_m$ is the largest root of $X_m$.
    Then we will show some properties concerning these polynomials.
    In particular, we will see that, for all $m \geq 0$, $Y_m$ is a linear combination with non-negative coefficients
    of $X_0, X_1, \dots, X_{2m-1}$.
    \newline

    It follows that, if $\int_{-L}^{L} X_m(x) d \varv(x) \geq 0$, then $\int_{-L}^{L} Y_m(x) d \varv(x) \geq 0$.
    Since $Y_m(x) \leq 0$ when $x \leq \alpha_m$, and $\alpha_m > 2 - \epsilon$ for $m$ large enough,
    it can be proved that the measure of $\varv$ in the interval $[2-\epsilon, L]$ is strictly positive.
    Finally, we see that actually $\varv[2 - \epsilon, L] \geq C(\epsilon) > 0$, by using a compactness argument in the
    weak topology.
    \newline

    The next step is to prove that $\varv$ effectively satisfies $\int_{-L}^{L} X_m(x) d \varv(x) \geq 0$,
    which follows the definition of $\varv$ and Dirac measure:
    $$
    \int_{-L}^L U_m(\frac{x}{2}) d\varv(x) =
        \frac{1}{n} \sum_{j=0}^{n-1} \int_{-L}^L U_m\left(\frac{x}{2}\right) d \delta_{\frac{\mu_j}{\sqrt{k-1}}} =
        \frac{1}{n} \sum_{j=0}^{n-1} U_m\left( \frac{\mu_j}{\sqrt{k-1}} \right)
    $$
    The assumption is satisfied, therefore there exists $C = C(\epsilon, k) > 0$ such that $\varv[2 - \epsilon] \geq C$.
    %Using equation~\eqref{eq:relation_A_U}, we show that, for an appropriate choice of $L \geq 2$, $\varv$ satisfies:
    %$$\int_{-L}^L U_m\left(\frac{x}{2}\right) d\varv \geq 0$$
    %for all $m$.
    %Finally, we will see that this condition is sufficient to prove that the measure $\varv$ has a positive support in the interval $[2-\epsilon, L]$.
    %In other words,
    %$$\varv[2-\epsilon, L] \geq C$$
    %where $C$ is a constant that only depends on $\epsilon$ and $L$, and is strictly positive.
    This proves the main result.
    \newline

    Additionally, we show a similar conclusion for the negative eigenvalues of the graph, generalizing an earlier result.

\end{document}